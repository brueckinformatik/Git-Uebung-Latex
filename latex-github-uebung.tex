\documentclass[10pt]{article}

\usepackage[utf8]{inputenc}
\usepackage[T1]{fontenc}
\usepackage[ngerman]{babel}
\usepackage{ tikz}
\usepackage[autostyle, german=quotes]{csquotes}
\usetikzlibrary{calc, positioning}
\title{Versionsverwaltung von \LaTeX{}}
\author{Arne Brück}
\date{05.05.22}

\begin{document}
\maketitle 

\section{Zusammenfassung}
Wer von Git hört, denkt häufig an GitHub: Dies ist der Ort, der die Hälfte der
Software der Welt enthält und von dem man sich die Software
kostenlos herunterladen kann.

Git ist aber unabhängig von GitHub, es ist ein Programm zur
Versionsverwaltung. Es ermöglicht auf einfache Weise Sicherheitskopien
des Projektes zu erstellen und zu späteren Zeitpunkten auf alle
früheren Veränderungen zugreifen zu können. Dies soll anhand dieses
Projektes verdeutlicht werden. Zusätzlich kann im Anschluss über
GitHub das Projekt gesichert werden oder mit anderen Menschen zusammen
komfortabel an diesem gearbeitet werden.

\section{Grundlagen}

\begin{tikzpicture}
  \node (AV) [thick, draw, rounded corners, rectangle, text width = 5em, text
  centered] {Arbeits\-verzeichnis};
  \node (Stage) [thick, draw, rounded corners, rectangle, text width = 5em, text
  centered, right = of AV ] {Stage};
  \node (Rep) [thick, draw, rounded corners, rectangle, text width = 5em, text
  centered, right  = of Stage] {Repository};
  \node (GitHub) [thick, draw, rounded corners, rectangle, text width = 5em, text
  centered, right  = of Rep, dashed] {GitHub};
  \draw [ultra thick, ->, shorten >=4pt, shorten <=2pt, dashed] (Rep) to [out = 45, in = 135] (GitHub);
  \draw [ultra thick, ->, shorten >=4pt, shorten <=2pt, dashed] (GitHub) to [out = 225, in = -45]  (Rep);
  \draw [ultra thick, ->, shorten >=4pt, shorten <=2pt] (AV) to [out = 45, in = 135] (Stage);
  \draw [ultra thick, ->, shorten >=4pt, shorten <=2pt] (Stage) to [out = 225, in = -45]  (AV);
  \draw [ultra thick, ->, shorten >=4pt, shorten <=2pt] (Stage) to [out = 45, in = 135] (Rep);
  \draw [ultra thick, ->, shorten >=4pt, shorten <=2pt] (Rep) to [out = 225, in = -45]  (Stage);
\end{tikzpicture}

Das \textbf{Arbeitsverzeichnis} ist das normaler Verzeichnis, in dem auf dem
Computer die Dateien gespeichert werden.

Das \textbf{Repository} (Repo) ist der Inhalt des Arbeitsverzeichnis in allen
gespeicherten Varianten. Wurde im Arbeitsverzeichnis beispielsweise
eine Datei irrtümlicherweise gelöscht, kann ein älterer Stand des
Arbeitsverzeichnisses wiederhergestellt werden, um die Datei
wiederzubekommen. Hierfür muss natürlich vorher der Inhalt des
Arbeitsverzeichnisses im Repo gespeichert werden. Dies erfolgt
gewöhnlich immer, wenn eine Sinneinheit abgeschlossen worden ist. Hier
zum Beispiel vor und nach Erstellen des Diagramms. Das Speichern im
Repo wird als \textsl{Commit} bezeichnet.

Die \textbf{Stage} steht zwischen Arbeitsverzeichnis und Repository.
Ein Commit ist immer ein wesentlicher Schritt, der genau dokumentiert
wird und für alle anderen Menschen auf Ewigkeit sichtbar ist. Daher
werden alle Änderungen zuerst auf die Stage (Bühne) gesetzt und vor
dem Commit gründlich geprüft. Idealerweise sollte das folgende Commit
nicht 1 Minute später mit dem Kommentar \textsl{Diese Datei hatte ich
  vergessen} erfolgen.

Die Stage ist also ein Bereich in dem alles gesammelt und überprüft
wird, was in der Zukunft über einen Commit im  Repository registriert wird.



\section{Erkärung einiger Befehle}
\begin{itemize}
\item \textbf{git init:} Erster Befehl in einem Arbeitsverzeichnis,
  erstellt das noch leere Repository.
\item \textbf{git add . :} Setzt das gesamte Arbeitsverzeichnis auf
  die Stage. Statt dem Punkt (steht für das Arbeitsverzeichnis) könnte
  auch der Name einer Datei angegeben werden.
\item \textbf{git commit -m \enquote{Kommentar zum Commit}:} Alles auf
  der Stage wird im Repository mit dem Kommentar gesichert.
\item \textbf{git status -s:} Gibt eine Kurzübersicht über die Dateien im Arbeitsverzeichnis, der Stage und dem Repo. 

\end{itemize}

\section{Aufgaben}

\begin{enumerate}
  
\item Klonen Sie sich dieses Projekt mit dem Befehl: \linebreak \textsl{git clone https://github.com/brueckinformatik/Git-Uebung-Latex.git}\\
Legen Sie hierfür ein geeignetes Arbeitsverzeichnis an.
\item Erstellen Sie in Ihrem GitHub-Account ein leeres Repository mit dem Namen: Git-Uebung-Latex.git. Folgen Sie der angezeigten Anleitung und \textsl{pushen} Sie das geklonte Repository in Ihr GitHub-Account.
\item Bearbeiten Sie die folgenden Aufgaben und erstellen Sie nach jeder Aufgabe einen \textsl{Commit/Push}.
\item Schreiben Sie unter \textsl{Erklärung einiger Befehle} einen neuen Eintrag für den \textsl{git status}-Befehl ohne die Option \textsl{s}.
\item Schreiben Sie weitere Einträge für die git-Befehle: \textsl{config, clone, remote, checkout und push}.
  
\item Der Worstcase ist eingetreten, Sie haben aus Versehen a) den kompletten Text dieser Datei gelöscht und gespeichert, b) zusätzlich alles auf die Stage gestellt, c) zusätzlich ein Commit erstellt, d) zusätzlich einen Push durchgeführt. Erstellen Sie einen neuen Abschnitt in dem Dokument (Name z.B. Rettungsaktionen) und notieren Sie Ihre Lösungen für die Varianten.
\end{enumerate}

\end{document}

%%% Local Variables:
%%% mode: latex
%%% TeX-master: t
%%% End:
